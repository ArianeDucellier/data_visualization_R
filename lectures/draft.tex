% !TEX encoding = UTF-8 Unicode
\documentclass{beamer}

\usepackage{color}
\usepackage{gensymb}
\usepackage{hyperref}
\usepackage{textcomp}

\usetheme{Warsaw}

\title[The basics of visualization]{Data Visualization with R - The basics of visualization}
\author{Ariane Ducellier}
%\date{}

\begin{document}

	\begin{frame}
		\titlepage
	\end{frame}

	\begin{frame}


data Visualization

Different kind of variables

Continuous

Discrete

Categorical: Nominal
                    Ordinal
                    Logical
                    
Type of visualization: One-dimensional data, relationship between two one-dimensional data
          
One-dimensional object: histogram, barchart
Two-dimensional objects: bar chart, box plot, line chart, scatter plot

Histogram: One-dimensional, continuous data

	\end{frame}

	\begin{frame}
		\frametitle{Example: World fertility rates}

		From the functional art.
		No illustrations from book.
	\end{frame}

	\begin{frame}
		\frametitle{Example: Brazil armed forces}

		From the functional art.
		How to compare neighbors with Brazil

	\end{frame}

	\begin{frame}
		\frametitle{Example: Spain unemployment rate}

		From the functional art.
		How to compare regions with each other

	\end{frame}
	
\end{document}
