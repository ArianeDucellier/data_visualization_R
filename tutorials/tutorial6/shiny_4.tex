% !TEX encoding = UTF-8 Unicode
\documentclass{beamer}

\usepackage{color}
\usepackage{fancyvrb}
\usepackage{gensymb}
\usepackage{hyperref}
\usepackage{listings}
\usepackage{textcomp}
\usepackage{tikz}

\usetikzlibrary{arrows,positioning,shapes,shapes.multipart} 
\tikzstyle{every picture}+=[remember picture]

\definecolor{mygreen}{rgb}{0.88,0.95,0.88}

\usetheme{Warsaw}

\title[Data Visualization with R - Shiny tutorial]{Data Visualization with R \\ Shiny tutorial}
\author{Ariane Ducellier}
\date{University of Washington - Fall 2025}

\begin{document}

	\begin{frame}
		\titlepage
	\end{frame}

	\begin{frame}[fragile]
		\frametitle{Improving the UI - Using shiny themes}
		
		\begin{exampleblock}{}
		\begin{BVerbatim}
library(shinythemes)
fluidpage(theme=shinytheme("darkly"),
...)
		\end{BVerbatim}
		\end{exampleblock}{}

		\vspace{2em}

		If you want the user to be able to change the theme:
		
		\begin{exampleblock}{}
		\begin{BVerbatim}
library(shinythemes)
fluidpage(theme=shinytheme("darkly"),
          themeSelector(),
...)
		\end{BVerbatim}
		\end{exampleblock}{}

		\vspace{2em}

		See a list of themes here: \href{http://rstudio.github.io/shinythemes/}{http://rstudio.github.io/shinythemes/}
 	
	\end{frame}

	\begin{frame}[fragile]
		\frametitle{Improving the UI - Using the grid layout}
		
		\begin{exampleblock}{}
		\begin{BVerbatim}
fluidPage(title="...",
  fluidRow(
    column(6,
      wellPanel(
        sliderInput( ... ))),
    column(6, ... ))
  hr(),
  ...
)      
		\end{BVerbatim}
		\end{exampleblock}{}

		\vspace{1em}

		The sum of the  widths of the columns must be 12. \verb|wellPanel| creates a panel around the slider. \verb|hr()| creates a horizontal rule to break the screen.

	\end{frame}

	\begin{frame}[fragile]
		\frametitle{Downloading plots}

		In the file \verb|ui.R|:
		
		\begin{exampleblock}{}
		\begin{BVerbatim}
downloadButton("downloadPlot",
               label = "Download plot")
		\end{BVerbatim}
		\end{exampleblock}{}

		In the file \verb|server.R|:
		
		\begin{exampleblock}{}
		\begin{BVerbatim}
thePlot <- reactive( ... code to make plot ... )
output$downloadPlot <- downloadHandler(
  filename <- function(){"filename"},
  content <- function(file){
    png(file, width=980, height=400, ... )
    iris.plot <- thePlot()
    print(iris.plot)
    dev.off()
  },
  contentType = "image/png"
)
		\end{BVerbatim}
		\end{exampleblock}{}

	\end{frame}

	\begin{frame}[fragile]
		\frametitle{Downloading data}

		In the file \verb|ui.R|:
		
		\begin{exampleblock}{}
		\begin{BVerbatim}
downloadButton("downloadData",
               label = "Download data")
		\end{BVerbatim}
		\end{exampleblock}{}

		\vspace{1em}

		In the file \verb|server.R|:
		
		\begin{exampleblock}{}
		\begin{BVerbatim}
theData <- reactive( ... code to produce data ... )
output$downloadData <- downloadHandler(
  filename = function(){"iris.csv"},
  content <- function(file){
    write.csv(theData(), file)
  },
  contentType = "text/csv"
)
		\end{BVerbatim}
		\end{exampleblock}{}

	\end{frame}

	\begin{frame}[fragile]
		\frametitle{Interactive plots - Click points}

		In the file \verb|ui.R|:
		
		\begin{exampleblock}{}
		\begin{BVerbatim}
plotOutput("plot", click = "plot_click"),
                 tableOutput("plot_clickedpoints")
		\end{BVerbatim}
		\end{exampleblock}{}

		\vspace{1em}

		In the file \verb|server.R|:
		
		\begin{exampleblock}{}
		\begin{BVerbatim}
output$plot_clickedpoints <- renderTable({
  res <- nearPoints(iris,
                    input$plot_click,
                    "Sepal.Length",
                    "Sepal.Width")
  if (nrow(res) == 0)
    return()
  res
})
		\end{BVerbatim}
		\end{exampleblock}{}

	\end{frame}

	\begin{frame}[fragile]
		\frametitle{Interactive plots - Hover over plot}

		In the file \verb|ui.R|:
		
		\begin{exampleblock}{}
		\begin{BVerbatim}
plotOutput("plot",
           hover = hoverOpts(id = "plot_hover",
                             delayType = "throttle")
)
		\end{BVerbatim}
		\end{exampleblock}{}

		\vspace{1em}

		In the file \verb|server.R|:
		
		\begin{exampleblock}{}
		\begin{BVerbatim}
output$plot_hoverinfo <- renderPrint({
  cat("Hover (throttled):\n")
  str(input$plot_hover)
})  
		\end{BVerbatim}
		\end{exampleblock}{}

	\end{frame}

	\begin{frame}[fragile]
		\frametitle{Sharing with Gist}

		Go to \href{https://gist.github.com/}{https://gist.github.com/}.

		\vspace{1em}

		If you have a GitHub account, you should have on account on Gist too.

		\vspace{1em}

		Create a project with a description, an \verb|ui.R| and a \verb|server.R| files.

		\vspace{1em}

		Get the URL of your project.

		\vspace{1em}

		In RStudio, run:
		
		\begin{exampleblock}{}
		\begin{BVerbatim}
library(shiny)
runGist("https://gist.github.com/MyName/identifier")
		\end{BVerbatim}
		\end{exampleblock}{}

	\end{frame}

	\begin{frame}[fragile]
		\frametitle{Sharing with GitHub}

		On GitHub, create a repository with your dataset, and the \verb|ui.R| and \verb|server.R| files.

		\vspace{1em}

		In rStudio, run:
		
		\begin{exampleblock}{}
		\begin{BVerbatim}
library(shiny)
runGitHub("repository_name", "user_name")
		\end{BVerbatim}
		\end{exampleblock}{}

	\end{frame}

	\begin{frame}[fragile]
		\frametitle{Sharing through Shinyapps.io}

		Create a free account on \href{https://www.shinyapps.io/}{https://www.shinyapps.io/}.

		\vspace{1em}

		In RStudio, install the package \verb|rsconnect|.

		\vspace{1em}

		When creating your account, \verb|Shinyapps.io| will ask you to set your \verb|Shinyapps.io| account information on RStudio by running:
		
		\begin{exampleblock}{}
		\begin{BVerbatim}
rsconnect::setAccountInfo(name='yourname',
token='some_token', secret='some_secret')
		\end{BVerbatim}
		\end{exampleblock}{}

		To deploy your application on \verb|Shinyapps.io|, run on RStudio:

		\begin{exampleblock}{}
		\begin{BVerbatim}
library(rsconnect)
rsconnect::deployApp("/path/your_path_to_your_app")
		\end{BVerbatim}
		\end{exampleblock}{}

		Check on \verb|Shinyapps.io| the URL of your application:

		\href{https://arianeducellier.shinyapps.io/magnitudes/}{https://arianeducellier.shinyapps.io/magnitudes/}

	\end{frame}

\end{document}
