% !TEX encoding = UTF-8 Unicode
\documentclass{beamer}

\usepackage{color}
\usepackage{fancyvrb}
\usepackage{gensymb}
\usepackage{hyperref}
\usepackage{listings}
\usepackage{textcomp}
\usepackage{tikz}

\usetikzlibrary{arrows,positioning,shapes,shapes.multipart} 
\tikzstyle{every picture}+=[remember picture]

\definecolor{mygreen}{rgb}{0.88,0.95,0.88}

\usetheme{Warsaw}

\title[Data Visualization with R - Shiny tutorial]{Data Visualization with R \\ Shiny tutorial}
\author{Ariane Ducellier}
\date{University of Washington - Fall 2024}

\begin{document}

	\begin{frame}
		\titlepage
	\end{frame}

	\begin{frame}[fragile]
		\frametitle{Simple layouts}
		
		Left-to-right and top-to-bottom. The elements reorder themselves when resizing the window.
		\begin{exampleblock}{}
		\begin{BVerbatim}
flowLayout( ... )
		\end{BVerbatim}
		\end{exampleblock}{}

		\vspace{1em}

		Top-to-bottom
		\begin{exampleblock}{}
		\begin{BVerbatim}
verticalLayout( ... )
		\end{BVerbatim}
		\end{exampleblock}{}

		\vspace{1em}

		Left-to-right with manually set widths
		\begin{exampleblock}{}
		\begin{lstlisting}
splitLayout(cellWidths = c("25%", "75%"),
            ... ),
		\end{lstlisting}
		\end{exampleblock}{}

	\end{frame}

	\begin{frame}[fragile]
		\frametitle{Complete layouts}
		
		Side bar and main panel
		\begin{exampleblock}{}
		\begin{BVerbatim}
fluidpage(
  sidebarLayout(sidebarPanel, mainPanel, position))
)
		\end{BVerbatim}
		\end{exampleblock}{}

		\vspace{1em}

		Rows and columns. The sum of the widths of the columns must be equal to 12.
		\begin{exampleblock}{}
		\begin{BVerbatim}
fluidpage(
  fluidrow(
    column(width=4, ...),
    column(width=4, ...), ... ))
		\end{BVerbatim}
		\end{exampleblock}{}

	\end{frame}

	\begin{frame}[fragile]
		\frametitle{Tabs}

		We can show multiple frames on the screen and let the user select one. The processing of the data is only carried out for the currently selected tab.

		\vspace{1em}
		
		\begin{exampleblock}{}
		\begin{BVerbatim}
tabsetPanel(
  tabPanel("Slider", sliderInput("slider",
                                 "Slider",
                                 min = 1,
                                 max = 100,
                                 value = 50)),
  tabPanel("Text input", textInput("text", "Text")),
  tabPanel("Table", tableOutput("table")
)
		\end{BVerbatim}
		\end{exampleblock}{}

	\end{frame}

	\begin{frame}[fragile]
		\frametitle{Complete layouts}

		Top level navigation bar and several tabs
		\begin{exampleblock}{}
		\begin{BVerbatim}
navbarPage(title, tabPanel)
		\end{BVerbatim}
		\end{exampleblock}{}

		\vspace{2em}

		Left navigation bar and several tabs
		\begin{exampleblock}{}
		\begin{BVerbatim}
fluidpage(
  navlistPanel(title, tabPanel)
)
		\end{BVerbatim}
		\end{exampleblock}{}

	\end{frame}

	\begin{frame}[fragile]
		\frametitle{Complete layouts}

		Combination of layouts
		\begin{exampleblock}{}
		\begin{BVerbatim}
fluidPage(
  fluidRow(
    column(width=4, ...), column(width=8, ...)),
  splitLayout( ... ),
  verticalLayout( ... )
)
		\end{BVerbatim}
		\end{exampleblock}{}

	\end{frame}

	\begin{frame}[fragile]
		\frametitle{Functions in ui.R and server.R}
		\begin{center}
		\begin{tabular}{ c c c }
		\verb|ui.R| & & \verb|server.R| \\
		\hline
		\\
		\verb|textOutput| & $\leftrightarrow$ & \verb|renderText| \\
		\\
		\verb|plotOutput| & $\leftrightarrow$ & \verb|renderPlot| \\
		\\
		\verb|tableOutput| & $\leftrightarrow$ & \verb|renderTable| \\
		\\
		\verb|dataTableOutput| & $\leftrightarrow$ & \verb|renderDataTable| \\
		\\
		\verb|leafletOutput| & $\leftrightarrow$ & \verb|renderLeaflet| \\ 
		\end{tabular}
		\end{center}

		\vspace{1em}

		The \verb|leaflet| package allows us to produce maps shown with \verb|leafletOutput| in the \verb|ui.R| and created with \verb|renderLeaflet| in the \verb|server.R| file.

	\end{frame}

	\begin{frame}[fragile]
		\frametitle{Tables - Basic Shiny}
		
		In \verb|ui.R|:
		\begin{exampleblock}{}
		\begin{BVerbatim}
tableOutput("textDisplay")
		\end{BVerbatim}
		\end{exampleblock}{}

		\vspace{1em}

		In \verb|server.R|
		\begin{exampleblock}{}
		\begin{BVerbatim}
output$textDisplay = renderTable({
  getMat = matrix(c( ... ), ncol = 2, byrow = TRUE)
  colnames(getMat) = c("Value", "Class")
  getMat
})
		\end{BVerbatim}
		\end{exampleblock}{}

	\end{frame}

	\begin{frame}[fragile]
		\frametitle{Tables - With package DT (DataTable)}
		
		In \verb|ui.R|:
		\begin{exampleblock}{}
		\begin{BVerbatim}
dataTableOutput("countryTable")
		\end{BVerbatim}
		\end{exampleblock}{}

		\vspace{1em}

		In \verb|server.R|
		\begin{exampleblock}{}
		\begin{BVerbatim}
output$countryTable = renderDataTable({
  dataTable(
    ... ,
    colnames = ... ,
    caption = ... ,
    filter = "top",
    options = list(pageLength = 15,
                   lengthMenu = c(10, 20, 50))
})
		\end{BVerbatim}
		\end{exampleblock}{}

	\end{frame}

	\begin{frame}[fragile]
		\frametitle{Hiding elements}
		
		Name the panels:
		\begin{exampleblock}{}
		\begin{BVerbatim}
tabsetPanel(id = "theTabs",
  tabPanel( ... , value = "trend"),
  ...
)
		\end{BVerbatim}
		\end{exampleblock}{}

		\vspace{1em}

		Add a condition to show an UI element only if a tab is selected:
		\begin{exampleblock}{}
		\begin{BVerbatim}
conditionalPanel(
  condition = "input.theTabs == ‘trend’",
  checkboxInput( ... )
)
		\end{BVerbatim}
		\end{exampleblock}{}

	\end{frame}

	\begin{frame}[fragile]
		\frametitle{Reactive objects}

		In the \verb|server.r| file, we filter the data using a reactive object:

		\vspace{1em}
		
		\begin{exampleblock}{}
		\begin{lstlisting}
theData = reactive({
  mapData %>%
    filter(year >= input$year)
})
		\end{lstlisting}
		\end{exampleblock}{}

		\vspace{1em}

		\begin{itemize}
			\item A reactive object changes when its input changes.
			\item When it runs, the output is cached.
			\item If it is called several times in an application, it will not run again if the inputs are unchanged.
		\end{itemize}

	\end{frame}



	\begin{frame}[fragile]
		\frametitle{Reactive user interfaces}
		
		In \verb|ui.R|:
		\begin{exampleblock}{}
		\begin{BVerbatim}
uiOutput("yearSelectorUI")
		\end{BVerbatim}
		\end{exampleblock}{}

		\vspace{1em}

		In \verb|server.R|
		\begin{exampleblock}{}
		\begin{BVerbatim}
output$yearSelectorUI = renderUI(
  selectedYears = ...
  selectInput( ... , selectedYears)
})
		\end{BVerbatim}
		\end{exampleblock}{}

		\vspace{1em}

		When the value in \verb|selectedYears| change, the choice of years in the widget will also change.

	\end{frame}

\end{document}
