% !TEX encoding = UTF-8 Unicode
\documentclass{beamer}

\usepackage{color}
\usepackage{fancyvrb}
\usepackage{gensymb}
\usepackage{hyperref}
\usepackage{listings}
\usepackage{textcomp}
\usepackage{tikz}

\usetikzlibrary{arrows,positioning,shapes,shapes.multipart} 
\tikzstyle{every picture}+=[remember picture]

\definecolor{mygreen}{rgb}{0.88,0.95,0.88}

\usetheme{Warsaw}

\title[Data Visualization with R - Dash tutorial]{Data Visualization with R \\ Dash tutorial}
\author{Ariane Ducellier}
\date{University of Washington - Fall 2023}

\begin{document}

	\begin{frame}
		\titlepage
	\end{frame}

	\begin{frame}
		\frametitle{What is Dash?}

	\end{frame}

	\begin{frame}[fragile]
		\frametitle{Installing Dash}

		In RStudio, run:

		\begin{exampleblock}{}
		\begin{BVerbatim}
install.packages("remotes")
remotes::install_github("plotly/dashR",
                        upgrade = "always")
		\end{BVerbatim}
		\end{exampleblock}{}

	\end{frame}

	\begin{frame}[fragile]
		\frametitle{About Dash syntax}

		The three bits of codes are equivalent:

		\begin{exampleblock}{}
		\begin{BVerbatim}
app <- dash_app()
app %>% set_layout(
app %>% run_app()
		\end{BVerbatim}
		\end{exampleblock}{}

		\begin{exampleblock}{}
		\begin{BVerbatim}		
dash_app() %>%
  set_layout(...) %>%
  run_app()
		\end{BVerbatim}
		\end{exampleblock}{}

		\begin{exampleblock}{}
		\begin{BVerbatim}
app <- dash_app()
set_layout(app, ...)
run_app(app)
		\end{BVerbatim}
		\end{exampleblock}{}

	\end{frame}

%	\begin{frame}[fragile]
%		\frametitle{Examples}
%
%		Let us run several examples:
%
%		\vspace{2em}
%
%		\begin{exampleblock}{}
%		\begin{BVerbatim}
%		library(shiny)
%		\end{BVerbatim}
%		\end{exampleblock}{}
%
%		\vspace{2em}
%
%		\begin{exampleblock}{}
%		\begin{BVerbatim}
%		runExample("08_html")
%		\end{BVerbatim}
%		\end{exampleblock}{}
%
%		\vspace{2em}
%
%		\begin{exampleblock}{}
%		\begin{BVerbatim}
%		runExample("01_hello")
%		\end{BVerbatim}
%		\end{exampleblock}{}
%
%	\end{frame}

\end{document}
