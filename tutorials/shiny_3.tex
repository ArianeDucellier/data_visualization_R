% !TEX encoding = UTF-8 Unicode
\documentclass{beamer}

\usepackage{color}
\usepackage{fancyvrb}
\usepackage{gensymb}
\usepackage{hyperref}
\usepackage{listings}
\usepackage{textcomp}
\usepackage{tikz}

\usetikzlibrary{arrows,positioning,shapes,shapes.multipart} 
\tikzstyle{every picture}+=[remember picture]

\definecolor{mygreen}{rgb}{0.88,0.95,0.88}

\usetheme{Warsaw}

\title[Data Visualization with R - Shiny tutorial]{Data Visualization with R \\ Shiny tutorial}
\author{Ariane Ducellier}
\date{University of Washington - Fall 2023}

\begin{document}

	\begin{frame}
		\titlepage
	\end{frame}

	\begin{frame}[fragile]
		\frametitle{Hiding elements}
		
		Name the panels:
		\begin{exampleblock}{}
		\begin{BVerbatim}
tabsetPanel(id = "theTabs",
tabPanel( ... , value = "trend"),
  ...
)
		\end{BVerbatim}
		\end{exampleblock}{}

		\vspace{1em}

		Add a condition to show an UI element only if a tab is selected:
		\begin{exampleblock}{}
		\begin{BVerbatim}
conditionalpanel(
  condition = "input.theTabs == ‘trend’",
  checkboxInput( ... )
)
		\end{BVerbatim}
		\end{exampleblock}{}

	\end{frame}

	\begin{frame}[fragile]
		\frametitle{Tables - Basic Shiny}
		
		In \verb|ui.R|:
		\begin{exampleblock}{}
		\begin{BVerbatim}
tableOutput("textDisplay")
		\end{BVerbatim}
		\end{exampleblock}{}

		\vspace{1em}

		In \verb|server.R|
		\begin{exampleblock}{}
		\begin{BVerbatim}
output$textDisplay = renderTable({
  getMat = matrix(c( ... ), ncol = 2, byrow = TRUE)
  colnames(getMat) = c("Value", "Class")
  getMat
})
		\end{BVerbatim}
		\end{exampleblock}{}

	\end{frame}

	\begin{frame}[fragile]
		\frametitle{Tables - With package DT (DataTable)}
		
		In \verb|ui.R|:
		\begin{exampleblock}{}
		\begin{BVerbatim}
dataTableOutput("countryTable")
		\end{BVerbatim}
		\end{exampleblock}{}

		\vspace{1em}

		In \verb|server.R|
		\begin{exampleblock}{}
		\begin{BVerbatim}
output$countryTable = renderDataTable({
  dataTable(
    ... ,
    colnames = ... ,
    caption = ... ,
    filter = "top",
    options = list(pageLength = 15,
                   lengthMenu = c(10, 20, 50))
})
		\end{BVerbatim}
		\end{exampleblock}{}

	\end{frame}

	\begin{frame}[fragile]
		\frametitle{Reactive user interfaces}
		
		In \verb|ui.R|:
		\begin{exampleblock}{}
		\begin{BVerbatim}
uiOutput("yearSelectorUI")
		\end{BVerbatim}
		\end{exampleblock}{}

		\vspace{1em}

		In \verb|server.R|
		\begin{exampleblock}{}
		\begin{BVerbatim}
output$yearSelectorUI = renderUI(
  selectedYears = ...
  selectInput( ... , selectedYears)
})
		\end{BVerbatim}
		\end{exampleblock}{}

		\vspace{1em}

		When the value in \verb|selectedYears| change, the choice of years in the widget will also change.

	\end{frame}

	\begin{frame}[fragile]
		\frametitle{Progress bar}

		If some computation in \verb|server.R| can take a long time, it is useful to wrap the corresponding code inside the Shiny \verb|withProgress()| function.

		\vspace{1em}

		In \verb|server.R|
		\begin{exampleblock}{}
		\begin{BVerbatim}
withProgress(message = ... ,
  detail = ‘ ... ‘, value = 0,
  ... function code ...
  incProgress(1/3)
  ... function code ...
  incProgress(1/3)
  ... function code ...
  incProgress(1/3)
  ... function code ...
})
		\end{BVerbatim}
		\end{exampleblock}{}

	\end{frame}

\end{document}
