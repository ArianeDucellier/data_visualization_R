% !TEX encoding = UTF-8 Unicode
\documentclass{beamer}

\usepackage{color}
\usepackage{fancyvrb}
\usepackage{gensymb}
\usepackage{hyperref}
\usepackage{textcomp}

\usetheme{Warsaw}

\title[Ggplot2 tutorial]{Data Visualization with R - Ggplot2 tutorial}
\author{Ariane Ducellier}
%\date{}

\begin{document}

	\begin{frame}
		\titlepage
	\end{frame}

	\begin{frame}
		Something about ggplot2 and why we are learning it
	\end{frame}

	\begin{frame}
		\frametitle{Main concepts of Ggplot2}

		Ggplot2 is made of geometric objects (e.g, lines, bars, points) that are used to visualize data.

		Examples of one-dimensional objects for one-dimensional data:
		\begin{itemize}
			\item histogram
			\item bar-chart
		\end{itemize}

		Two-dimensional objects for the relationship between two variables:
	\end{frame}

	\begin{frame}
		\frametitle{Main concepts of Ggplot2}
		
		To the geometric objects, we are going to add aesthetics:
		\begin{itemize}
			\item Coordinates scale,
			\item Fonts,
			\item Colors...
		\end{itemize}

		The aesthetics are described using the grammar of graphics.
	\end{frame}

	\begin{frame}[fragile]
		\frametitle{Example: Histograms}

		Built-in R graphics package:

		\begin{exampleblock}{}
		\begin{center}
		\begin{BVerbatim}
		hist(airquality$Temp)
		\end{BVerbatim}
		\end{center}
		\end{exampleblock}{}

		Quick plot using ggplot2:

		\begin{exampleblock}{}
		\begin{center}
		\begin{BVerbatim}
		qplot(airquality$Temp)
		\end{BVerbatim}
		\end{center}
		\end{exampleblock}{}

	\end{frame}

	\begin{frame}[fragile]
		\frametitle{Ggplot2 command structure}

		\begin{exampleblock}{}
		\begin{center}
		\begin{BVerbatim}
		ggplot(airquality, aes(x=Temp))
		\end{BVerbatim}
		\end{center}
		\end{exampleblock}{}

dataset, aes = describe the variables from the dataset that we want to visualize and their qualities)

		This command does not plot anything.

	\end{frame}

	\begin{frame}[fragile]
		\frametitle{Ggplot2 command structure}

		We need to add a command to explain the kind of object that we want to plot:

		\begin{exampleblock}{}
		\begin{center}
		\begin{BVerbatim}
		ggplot(airquality, aes(x=Temp)) + geom_histogram()
		\end{BVerbatim}
		\end{center}
		\end{exampleblock}{}
		
	\end{frame}

	\begin{frame}[fragile]
		\frametitle{Bar plots}

		We can use bar plots to visualize one categorical variable:

		\begin{exampleblock}{}
		\begin{center}
		\begin{BVerbatim}
		df_desc <- read.csv(« ../data/
		    historical-hourly-weather-data/
		    weather_description.csv »)
		ggplot(df_desc, aes(x=Vancouver)) + geom_bar()
		\end{BVerbatim}
		\end{center}
		\end{exampleblock}{}

		The height of the bar is proportional to the number of cases in each group.
		
	\end{frame}

	\begin{frame}[fragile]
		\frametitle{Bar plots}

		Or a combination of a categorical variable and a continuous variable:

		\begin{exampleblock}{}
		\begin{center}
		\begin{BVerbatim}
		ggplot(RetailSales, aes(x=Month, y=Sales)) + 
		geom_bar(stat=« identity »)
		\end{BVerbatim}
		\end{center}
		\end{exampleblock}{}

		Using \verb|stat = « identity »| tells ggplot2 to sum the values for each group (Month) and plot bars proportional to the sums.
		
	\end{frame}

\end{document}
