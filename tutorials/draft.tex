% !TEX encoding = UTF-8 Unicode
\documentclass{beamer}

\usepackage{color}
\usepackage{gensymb}
\usepackage{hyperref}
\usepackage{textcomp}

\usetheme{Warsaw}

\title[Ggplot2 tutorial]{Data Visualization with R - Ggplot2 tutorial}
\author{Ariane Ducellier}
%\date{}

\begin{document}

	\begin{frame}
		\titlepage
	\end{frame}

	\begin{frame}

Tutorial ggplot2

ggplot2 is made of geometric objects (e.g, lines, bars, points) that are used to visualize data.

One-dimensional objects for one-dimensional data:

histogram
bar-chart

Two-dimensional objects for the relationship between two variables

Command line

ggplot ( dataset, as = describe the variables from the dataset that we want to visualize and their qualities)

do not plot anything

need to add + geom\_histogram() to explain the kind of object that you want to plot.

	\end{frame}

\end{document}
