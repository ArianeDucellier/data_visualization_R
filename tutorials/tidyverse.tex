% !TEX encoding = UTF-8 Unicode
\documentclass{beamer}

\usepackage{color}
\usepackage{fancyvrb}
\usepackage{gensymb}
\usepackage{hyperref}
\usepackage{textcomp}
\usepackage{tikz}

\usetikzlibrary{arrows,positioning,shapes,shapes.multipart} 
\tikzstyle{every picture}+=[remember picture]

\definecolor{mygreen}{rgb}{0.88,0.95,0.88}

\usetheme{Warsaw}

\title[Data Visualization with R - Tidyverse tutorial]{Data Visualization with R \\ Tidyverse tutorial}
\author{Ariane Ducellier}
\date{University of Washington - Fall 2023}

\begin{document}

	\begin{frame}
		\titlepage
	\end{frame}

	\begin{frame}[fragile]
		\frametitle{What is tidyverse?}

		A collection of R packages designed for data science.

		\vspace{1em}

		Basic packages:
		\begin{itemize}
			\item \textbf{ggplot2:} graphics
			\item \textbf{dplyr:} data manipulation
			\item \textbf{tidyr:} getting to tidy data
			\item \textbf{readr:} reading rectangular data (e.g. csv, tsv, fwf)
			\item \textbf{purrr:} working with functions and vectors
			\item \textbf{tibble:} a modern re-imagining of the data frame
			\item \textbf{stringr:} working with strings
			\item \textbf{forcats:} working with R factors to handle categorical variables
		\end{itemize}

		\vspace{1em}

		Additional packages associated to \verb|tidyverse| need to be installed and loaded separately to import data, wrangle data, program and model.		
		
	\end{frame}

	\begin{frame}
		\frametitle{Main concepts of data wrangling}

		\begin{itemize}
		\setlength{\itemsep}{1em}
			\item Understand.
			\item Format $\rightarrow$ Produce tidy data:
			\begin{itemize}
				\item Every column is a variable.
				\item Every row is an observation.
				\item Every cell is a single value.
			\end{itemize}
			\item Clean.
			\item Enrich.
			\item Validate.
			\item Analysis / Model $\rightarrow$ In our case, we are going to produce visuals to communicate information on the dataset to the viewer.
		\end{itemize}

	\end{frame}

	\begin{frame}
		\frametitle{Tibbles versus Data frames}

		\begin{itemize}
		\setlength{\itemsep}{1em}
			\item Tibbles do not change input variable types by default.
			\item Tibbles can have lists as columns.
			\item Tibbles can have non-standard variable names.
			\item Tibbles return another Tibble when slicing (and not a vector).
		\end{itemize}

	\end{frame}

	\begin{frame}[fragile]
		\frametitle{The pipe operator}

		The \verb|magrittr| package provides the $\%<>\%$ operator as a shortcut for modifying an object in place:

		\vspace{1em}

		\begin{exampleblock}{}
		\begin{center}
		\begin{BVerbatim}
		df_iris <- iris %>%
		  group_by(Species) %>%
		  summarize_if(is.numeric, mean) %>%
		  ungroup() %>%
		  gather(measure, value, -Species) %>%
		  arrange(value)
		\end{BVerbatim}
		\end{center}
		\end{exampleblock}{}

		=

		\begin{exampleblock}{}
		\begin{center}
		\begin{BVerbatim}
		df_iris <- group_by(iris, Species)
		df_iris <- summarize_if(df_iris, is.numeric, mean)
		df_iris <- ungroup(df_iris)
		df_iris <- gather(df_iris, measure, value, -Species)
		df_iris <- arrange(df_iris, value)
		\end{BVerbatim}
		\end{center}
		\end{exampleblock}{}

	\end{frame}

	\begin{frame}
%join

	\end{frame}

	\begin{frame}
%Part 2
%
%Mice for dealing with missing values
%rvest web scrapping
%httr and json for API

	\end{frame}

\end{document}
