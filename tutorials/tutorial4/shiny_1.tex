% !TEX encoding = UTF-8 Unicode
\documentclass{beamer}

\usepackage{color}
\usepackage{fancyvrb}
\usepackage{gensymb}
\usepackage{hyperref}
\usepackage{listings}
\usepackage{textcomp}
\usepackage{tikz}

\usetikzlibrary{arrows,positioning,shapes,shapes.multipart} 
\tikzstyle{every picture}+=[remember picture]

\definecolor{mygreen}{rgb}{0.88,0.95,0.88}

\usetheme{Warsaw}

\title[Data Visualization with R - Shiny tutorial - Part 1]{Data Visualization with R \\ Shiny tutorial - Part 1}
\author{Ariane Ducellier}
\date{University of Washington - Fall 2025}

\begin{document}

	\begin{frame}
		\titlepage
	\end{frame}

	\begin{frame}
		\frametitle{What is Shiny?}

		\begin{itemize}
			\item R is an package that enables building interactive web applications that can execute R code on the backend.

			\vspace{1em}

			\item What can you do with Shiny?
			\begin{itemize}
				\item Host standalone applications on a webpage
				\item Embed interactive charts in R Markdown documents.
				\item Build dashboards.
				\item Perform any R calculation and display the results on the webpage or dashboard.
				\item Extend your Shiny applications with CSS themes, HTML widgets, and JavaScript actions.
			\end{itemize}
		\end{itemize}
		
	\end{frame}

	\begin{frame}[fragile]
		\frametitle{Examples}

		Let us run several examples:

		\vspace{2em}

		\begin{exampleblock}{}
		\begin{BVerbatim}
		library(shiny)
		\end{BVerbatim}
		\end{exampleblock}{}

		\vspace{2em}

		\begin{exampleblock}{}
		\begin{BVerbatim}
		runExample("08_html")
		\end{BVerbatim}
		\end{exampleblock}{}

		\vspace{2em}

		\begin{exampleblock}{}
		\begin{BVerbatim}
		runExample("01_hello")
		\end{BVerbatim}
		\end{exampleblock}{}

	\end{frame}

	\begin{frame}[fragile]
		\frametitle{Examples}

		UI part:

		\begin{exampleblock}{}
		\begin{BVerbatim}
ui <- fluidPage(
  titlePanel(...),
  sidebarLayout(
    sidebarPanel(
      sliderInput(
        ...
      )
    ),
    mainPanel(
      plotOutput(outputId="distplot")
    )
  )
)
		\end{BVerbatim}
		\end{exampleblock}{}

	\end{frame}

	\begin{frame}[fragile]
		\frametitle{Examples}

		Server part:

		\begin{exampleblock}{}
		\begin{BVerbatim}
server <- function(input, output) {
  output$distplot <- renderPlot({
    ...
  })
}
		\end{BVerbatim}
		\end{exampleblock}{}

		\vspace{2em}

		Creation of Shiny app:

		\begin{exampleblock}{}
		\begin{BVerbatim}
shinyApp(ui=ui, server=server)
		\end{BVerbatim}
		\end{exampleblock}{}
	
	\end{frame}

	\begin{frame}[fragile]
		\frametitle{R Markdown with interactive Shiny elements}

		\begin{exampleblock}{}
		\begin{BVerbatim}
Go to File >
      New File >
      R Markdown >
      Shiny
		\end{BVerbatim}
		\end{exampleblock}{}

		\vspace{2em}

		Fill the document with the code from \verb|tutorial_shiny_1.Rmd|.

		\vspace{2em}

		Click on \verb|Run Document|.
	
	\end{frame}

	\begin{frame}[fragile]
		\frametitle{Minimal example}

		We need the files \verb|ui.R| and \verb|server.R| that are kept within the same folder. \verb|ui.R| describe the user interface.

		\vspace{2em}		

		\begin{exampleblock}{}
		\begin{BVerbatim}
fluidPage(...,
    title = NULL, theme = NULL, lang = NULL)
		\end{BVerbatim}
		\end{exampleblock}{}

		indicates that we are going to use a fluid page layout with rows containing columns.

		\vspace{2em}

		\begin{exampleblock}{}
		\begin{BVerbatim}
titlePanel(title, windowTitle = title)
		\end{BVerbatim}
		\end{exampleblock}{}

		describes the title of the application.

	\end{frame}

	\begin{frame}[fragile]
		\frametitle{Minimal example}
		
		\begin{exampleblock}{}
		\begin{BVerbatim}
sidebarLayout(sidebarPanel,
              mainPanel,
              position = c("left", "right"),
              fluid = TRUE)
		\end{BVerbatim}
		\end{exampleblock}{}

		\vspace{2em}

		describe the general layout of the page, with:
		\begin{itemize}
			\item Inputs on the side (\verb|sidebarPanel|),
			\item Outputs in the middle (\verb|mainPanel|).
		\end{itemize}
	
	\end{frame}

	\begin{frame}[fragile]
		\frametitle{Minimal example}

		The panels contain input and output widgets:

		\vspace{1em}
		
		\begin{exampleblock}{}
		\begin{BVerbatim}
textInput(inputId = "comment",
          label,
          value = "",
          width = NULL,
          placeholder = NULL)
		\end{BVerbatim}
		\end{exampleblock}{}

		\vspace{1em}

		\begin{exampleblock}{}
		\begin{BVerbatim}
textOutput(outputId = "textDisplay",
           container = if (inline) span else div,
           inline = FALSE)
		\end{BVerbatim}
		\end{exampleblock}{}

		\vspace{1em}

		\verb|server.R| contains functions which use \verb|inputid|	as an input, and produce \verb|outputId| as an output.

	\end{frame}

	\begin{frame}[fragile]
		\frametitle{Minimal example}

		\verb|server.R| contains a function describing how to use the input from \verb|ui.R| to produce the outputs from \verb|ui.R|.

		\vspace{1em}
		
		\begin{exampleblock}{}
		\begin{BVerbatim}
function(input, output){
  output$textDisplay = renderText({...input$comment...
  })
 )
		\end{BVerbatim}
		\end{exampleblock}{}

		\vspace{1em}

		The \verb|function(input, output)| contains the reactive components of the application. For example:
		
		\begin{exampleblock}{}
		\begin{BVerbatim}
renderText(expr,
           env = parent.frame(),
           quoted = FALSE,
           func = NULL)
		\end{BVerbatim}
		\end{exampleblock}{}

	\end{frame}

	\begin{frame}[fragile]
		\frametitle{Run the minimal example}

		Set the working directory to the folder that contains \verb|ui.R| and \verb|server.R|,

		\vspace{1em}
		
		\begin{exampleblock}{}
		\begin{BVerbatim}
setwd("/Users/my_name/Documents/my_folder/")
		\end{BVerbatim}
		\end{exampleblock}{}

		\vspace{1em}

		load the Shiny package:

		\vspace{1em}
		
		\begin{exampleblock}{}
		\begin{BVerbatim}
library(shiny)
		\end{BVerbatim}
		\end{exampleblock}{}

		\vspace{1em}

		and run the application:

		\vspace{1em}
		
		\begin{exampleblock}{}
		\begin{BVerbatim}
runApp()
		\end{BVerbatim}
		\end{exampleblock}{}

	\end{frame}

	\begin{frame}[fragile]
		\frametitle{Various widgets}

		\begin{exampleblock}{}
		\begin{BVerbatim}
checkboxGroupInput(inputId, label, choices=NULL, ...)
		\end{BVerbatim}
		\end{exampleblock}{}

		\vspace{1em}

		\begin{exampleblock}{}
		\begin{BVerbatim}
checkboxInput(inputId, label, value=FALSE, ...)
		\end{BVerbatim}
		\end{exampleblock}{}

		\vspace{1em}

		\begin{exampleblock}{}
		\begin{BVerbatim}
dateInput(inputId, label, ...)
		\end{BVerbatim}
		\end{exampleblock}{}

		\vspace{1em}

		\begin{exampleblock}{}
		\begin{BVerbatim}
dateRangeInput(inputId, label, ...)
		\end{BVerbatim}
		\end{exampleblock}{}

		\vspace{1em}

		\begin{exampleblock}{}
		\begin{BVerbatim}
numericInput(inputId, label, value, ...)
		\end{BVerbatim}
		\end{exampleblock}{}

		\vspace{1em}

		\begin{exampleblock}{}
		\begin{BVerbatim}
radioButtons(inputId, label, choices=NULL, ...)
		\end{BVerbatim}
		\end{exampleblock}{}

	\end{frame}

	\begin{frame}[fragile]
		\frametitle{Various widgets}

		\begin{exampleblock}{}
		\begin{BVerbatim}
selectInput(inputId, label, choices, ...)
		\end{BVerbatim}
		\end{exampleblock}{}

		\vspace{1em}

		\begin{exampleblock}{}
		\begin{BVerbatim}
sliderInput(inputId, label, min, max, value, ...)
		\end{BVerbatim}
		\end{exampleblock}{}

		\vspace{1em}

		\begin{exampleblock}{}
		\begin{BVerbatim}
textInput(inputId, label, ...)
		\end{BVerbatim}
		\end{exampleblock}{}

		\vspace{1em}

		To see an example of how the widgets look like, type:

		\vspace{1em}

		\begin{exampleblock}{}
		\begin{BVerbatim}
library(shiny)
runGist(6571951)
		\end{BVerbatim}
		\end{exampleblock}{}

	\end{frame}

\end{document}
