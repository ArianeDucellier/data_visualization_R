% !TEX encoding = UTF-8 Unicode
\documentclass{beamer}

\usepackage{color}
\usepackage{fancyvrb}
\usepackage{gensymb}
\usepackage{hyperref}
\usepackage{listings}
\usepackage{textcomp}
\usepackage{tikz}

\usetikzlibrary{arrows,positioning,shapes,shapes.multipart} 
\tikzstyle{every picture}+=[remember picture]

\definecolor{mygreen}{rgb}{0.88,0.95,0.88}

\usetheme{Warsaw}

\title[Data Visualization with R - Shiny tutorial]{Data Visualization with R \\ Shiny tutorial}
\author{Ariane Ducellier}
\date{University of Washington - Fall 2023}

\begin{document}

	\begin{frame}
		\titlepage
	\end{frame}

	\begin{frame}[fragile]
		\frametitle{R Flexdashboard}

		\begin{exampleblock}{}
		\begin{BVerbatim}
Go to File >
      New File >
      R Markdown >
      From Template >
      Flex Dashboard
		\end{BVerbatim}
		\end{exampleblock}{}

		\vspace{2em}

		Click on \verb|Knit| to see the empty dashboard.
	
	\end{frame}

	\begin{frame}[fragile]
		\frametitle{R Flexdashboard}

		In the first R block, load the libraries and the data:
		
		\begin{exampleblock}{}
		\begin{BVerbatim}
library(flexdashboard)
library(tidyverse)
library(leaflet)
load("geocodedData.Rdata")
		\end{BVerbatim}
		\end{exampleblock}{}

		\vspace{2em}

		Change the names of the R Markdown headers and fill the R blocks with the code from \verb|dashboard1.Rmd|.

		\vspace{2em}

		Click on \verb|Knit| to see the final dashboard.
	
	\end{frame}

	\begin{frame}[fragile]
		\frametitle{Adding shiny to the flexdashboard}

		Modify the header by adding shiny and using a \verb|rows| orientation:
		
		\begin{exampleblock}{}
		\begin{BVerbatim}
title: "Flexdashboard 2"
runtime: shiny
		\end{BVerbatim}
		\end{exampleblock}{}

		\vspace{2em}

		We will add one sidebar column:
		
		\begin{exampleblock}{}
		\begin{BVerbatim}
Column {.sidebar}
		\end{BVerbatim}
		\end{exampleblock}{}

		\vspace{2em}

		We fill the R block with R shiny code to create a slider and a checkbox as done previously in \verb|ui.R|.

	\end{frame}

	\begin{frame}[fragile]
		\frametitle{Adding shiny to the flexdashboard}

		Create a simple row and a row with several tabs:
		
		\begin{exampleblock}{}
		\begin{BVerbatim}
Row
		\end{BVerbatim}
		\end{exampleblock}{}

		\begin{exampleblock}{}
		\begin{BVerbatim}
Row {.tabset}
		\end{BVerbatim}
		\end{exampleblock}{}

		\vspace{2em}

		We fill the R block with R shiny code to create plots as done previously in \verb|server.R|.

		\vspace{2em}

		In this case, the filtering is done for every block of R code. We cannot define a reactive object to filter the years.
 	
	\end{frame}


	\begin{frame}[fragile]
		\frametitle{Shiny dashboards}
		
		\begin{exampleblock}{}
		\begin{BVerbatim}
library(shinydashboard)
header <- dashboardHeader( )
sidebar <- dashboardSidebar()
body <- dashboardBody()
dashboardPage(header, sidebar, body,
  title = NULL,
  skin = c("blue", "black", "purple", "green", "red",
           "yellow"))
		\end{BVerbatim}
		\end{exampleblock}{}

	\end{frame}

	\begin{frame}[fragile]
		\frametitle{Adding a menu to the sidebar}
		
		\begin{exampleblock}{}
		\begin{BVerbatim}
sidebarMenu(id = NULL,
  menuItem("Name",
    icon = ... ,
    tabName = ... ,
    badgeLabel = ... ,
    badgeColor = ... ,
    ...
  ),
  sliderInput( ... )
)
		\end{BVerbatim}
		\end{exampleblock}{}

		\vspace{1em}

		\verb|tabName| will be referred to in the dashboard body to create the corresponding graph.

	\end{frame}

	\begin{frame}[fragile]
		\frametitle{Improving the UI - Adding a menu to the sidebar}
		
		\begin{exampleblock}{}
		\begin{BVerbatim}
tabItems(
  tabItem(tabName = ... ,
    fluidRow(
      box(width = 10,
          plotOutput("trend"),
          checkboxInput( ... )),
          box(width = 2, ... )
    ),
  )
)
		\end{BVerbatim}
		\end{exampleblock}{}

		\vspace{1em}

		\verb|tabName| corresponds to the value given in \verb|menuItem| in the sidebar.

	\end{frame}

	\begin{frame}[fragile]
		\frametitle{Improving the UI - Adding info boxes}

		In the file \verb|ui.R|:
		
		\begin{exampleblock}{}
		\begin{BVerbatim}
infoBoxOutput(width = 3, "infoYears")
		\end{BVerbatim}
		\end{exampleblock}{}

		\vspace{1em}

		In the file \verb|server.R|:
		
		\begin{exampleblock}{}
		\begin{BVerbatim}
output$infoYears = renderInfoBox({
  infoBox(title,
    value = NULL,
    icon = ... ,
    color = ... ,
    fill = ...
  )
)
		\end{BVerbatim}
		\end{exampleblock}{}

	\end{frame}


	\begin{frame}[fragile]
		\frametitle{Improving the UI - Adding icons}
		
		\begin{exampleblock}{}
		\begin{BVerbatim}
tabPanel("Trend",
          plotOutput("trend"),
          icon = icon("calendar"))
		\end{BVerbatim}
		\end{exampleblock}{}

		\begin{exampleblock}{}
		\begin{BVerbatim}
tabPanel("Summary",
         textOutput("summary"),
         icon = icon("user", lib = "glyphicon"))
		\end{BVerbatim}
		\end{exampleblock}{}

		\vspace{2em}

		See a list of icons here:
		\begin{itemize}
			\item \href{https://fontawesome.com/icons}{https://fontawesome.com/icons}
			\item \href{https://icons.getbootstrap.com/}{https://icons.getbootstrap.com/}
		\end{itemize}
 	
	\end{frame}

\end{document}
