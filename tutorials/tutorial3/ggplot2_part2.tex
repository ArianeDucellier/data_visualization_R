% !TEX encoding = UTF-8 Unicode
\documentclass{beamer}

\usepackage{color}
\usepackage{fancyvrb}
\usepackage{gensymb}
\usepackage{hyperref}
\usepackage{textcomp}
\usepackage{tikz}

\usetikzlibrary{arrows,positioning,shapes,shapes.multipart} 
\tikzstyle{every picture}+=[remember picture]

\definecolor{mygreen}{rgb}{0.88,0.95,0.88}

\usetheme{Warsaw}

\title[Data Visualization with R - Ggplot2 tutorial (part 2)]{Data Visualization with R \\ Ggplot2 tutorial (part 2)}
\author{Ariane Ducellier}
\date{University of Washington - Fall 2024}

\begin{document}

	\begin{frame}
		\titlepage
	\end{frame}

	\begin{frame}[fragile]
		\frametitle{Layers}

		Each plot can be thought as a separate variable, and the sum of the variables will make the final plot. You can define:

		\vspace{2em}

		\begin{exampleblock}{}
		\begin{BVerbatim}
p1 <- ggplot(df,
      aes(x=Electricity_consumption_per_capita))
p2 <- p1 + geom_histogram()
p3 <- p1 + geom_histogram(bins=15)
p4 <- p3 + xlab("Electricity consumption per capita")
		\end{BVerbatim}
		\end{exampleblock}{}

		\vspace{2em}

		and you can choose to plot \verb|p2|, \verb|p3|, or \verb|p4|.

	\end{frame}

	\begin{frame}[fragile]
		\frametitle{Scales}

		Scales \verb|scale_x_continuous| or \verb|scale_x_discrete| can be used to specify the axes. \verb|name|, \verb|limits|, \verb|breaks|, and \verb|labels| are the main parameters that can be adjusted.

		\vspace{2em}

		\begin{exampleblock}{}
		\begin{BVerbatim}
p1 <- ggplot(df, aes(x=gdp_per_capita))
p2 <- p1 + geom_histogram()
p3 <- p2 + scale_x_continuous(
           name="GDP per capita"),
           limits=c(0, 50000),
           breaks=seq(0, 40000, 4000),
           labels=c("0K", "4K", "8K", "12K", "16K",
           "20K",  "24K", "28K", "32K", "36K", "40K"))
		\end{BVerbatim}
		\end{exampleblock}{}

	\end{frame}

	\begin{frame}[fragile]
		\frametitle{Polar coordinates}

		You can define the coordinates with \verb|coord_cartesian| or \verb|coord_polar|:

		\vspace{2em}

		\begin{exampleblock}{}
		\begin{BVerbatim}
t <- seq(0, 360, by=15)
r <- 2
qplot(r, t) +
coord_polar(theta="y") +
scale_y_continuous(breaks=seq(0, 360, 30))
		\end{BVerbatim}
		\end{exampleblock}{}

	\end{frame}

	\begin{frame}[fragile]
		\frametitle{Facets}

		A Trellis display allows creating a plot for each group of a categorical variable:

		\vspace{2em}

		\begin{exampleblock}{}
		\begin{BVerbatim}
p <- ggplot(df,
     aes(x=gdp_per_capita,
         y=Electricity_consumption_per_capita)) +
     geom_point()
p + facet_grid(Country ~ .)
p + facet_grid(. ~ Country)
p + facet_wrap(~Country)
		\end{BVerbatim}
		\end{exampleblock}{}

		\vspace{2em}

		You can group subplots horizontally, vertically or wrapped together.

	\end{frame}

	\begin{frame}[fragile]
		\frametitle{Shapes and colors}

		You can change shape and color for the entire plot:

		\vspace{2em}

		\begin{exampleblock}{}
		\begin{BVerbatim}
ggplot(df, aes_string(x=var1, y=var2)) +
geom_point(color=2, shape=2)
		\end{BVerbatim}
		\end{exampleblock}{}

		\vspace{2em}

		Or assign a different shape and color for each group of a categorical variable:

		\vspace{2em}

		\begin{exampleblock}{}
		\begin{BVerbatim}
ggplot(df, aes_string(x=var1, y=var2)) +
geom_point(aes(color=Country, shape=Country))
		\end{BVerbatim}
		\end{exampleblock}{}

	\end{frame}

	\begin{frame}[fragile]
		\frametitle{Themes}

		Theme is used to change the non-data elements of the plot:

		\vspace{1em}

		\begin{center}
		\begin{tabular}{|l|l|l|}
		\hline
    		Theme & Type & Arguments \\ 
		\hline
		axis.title.x & element\_text & size, color, family, angle \\
		\hline
		axis.title.y & element\_text & size, color, family, angle \\
		\hline
		plot.background & element\_rect & fill, color, linewidth \\
		\hline
		panel.background & element\_rect & fill, fill, color, line width \\
		\hline
		panel.grid.major & element\_line & color, linetype, linewidth \\
		\hline
		\end{tabular}
		\end{center}

		\vspace{1em}

		Type \verb|?theme| to show all possible types of themes, their types and their arguments.

	\end{frame}

	\begin{frame}[fragile]
		\frametitle{Themes}

		You can add themes to the plot to customize the non-data elements:

		\vspace{1em}

		\begin{exampleblock}{}
		\begin{BVerbatim}
p1 <- ggplot(dfn, aes(Genre, WorldGross)) 
p2 <- p1+ geom_bar(aes(fill=LeadStudio), 
                       stat="Identity",
                       position="dodge")
p3 <- p2 + theme(axis.title.x=element_text(size=15),
                 axis.title.y=element_text(size=15),
plot.background=element_rect(fill="gray87"),
panel.background=element_rect(fill="beige"),
panel.grid.major=element_line(color="Gray",
                              linetype=1))
		\end{BVerbatim}
		\end{exampleblock}{}

	\end{frame}

	\begin{frame}[fragile]
		\frametitle{Themes}

		You can use predefined themes:

		\vspace{2em}

		\begin{exampleblock}{}
		\begin{BVerbatim}
p2 + theme_bw() + ggtitle("theme_bw()")
p2 + theme_classic() + ggtitle("theme_classic()")
p2 + theme_classic() + ggtitle("theme_gray()")
p2 + theme_minimal() + ggtitle("theme_minimal()")
		\end{BVerbatim}
		\end{exampleblock}{}

	\end{frame}

	\begin{frame}[fragile]
		\frametitle{Themes}

		You can also use define your own theme:

		\begin{exampleblock}{}
		\begin{BVerbatim}
mytheme <- theme(legend.title=element_blank(),
           legend.position="bottom",
           text=element_text(color="Blue"),
           axis.text=element_text(size=12,
                                  color="Red"),
           axis.title=element_text(size=rel(1.5)))
		\end{BVerbatim}
		\end{exampleblock}{}

		and use it for a single plot:

		\begin{exampleblock}{}
		\begin{BVerbatim}
p2 + mytheme + ggtitle("Changed Plot with my theme")
		\end{BVerbatim}
		\end{exampleblock}{}

		or for all the plots by placing it at the beginning of your code:

		\begin{exampleblock}{}
		\begin{BVerbatim}
theme_set(my_theme)
		\end{BVerbatim}
		\end{exampleblock}{}

	\end{frame}

	\begin{frame}[fragile]
		\frametitle{Themes}

		You can change the color palette.

		\vspace{2em}

		Type \verb|?scale_fill_brewer| to see all the color palettes available.

		\vspace{2em}

		\begin{exampleblock}{}
		\begin{BVerbatim}
p4 <- p2 + theme_bw() + ggtitle("theme_bw()")
p4 + scale_fill_brewer(palette="Spectral")
		\end{BVerbatim}
		\end{exampleblock}{}

	\end{frame}

\end{document}
