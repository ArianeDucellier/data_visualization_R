% !TEX encoding = UTF-8 Unicode
\documentclass{beamer}

\usepackage{color}
\usepackage{fancyvrb}
\usepackage{gensymb}
\usepackage{hyperref}
\usepackage{textcomp}
\usepackage{tikz}

\usetikzlibrary{arrows,positioning,shapes,shapes.multipart} 
\tikzstyle{every picture}+=[remember picture]

\definecolor{mygreen}{rgb}{0.88,0.95,0.88}

\usetheme{Warsaw}

\title[Data Visualization with R - Ggplot2 tutorial (part 2)]{Data Visualization with R \\ Ggplot2 tutorial (part 2)}
\author{Ariane Ducellier}
\date{University of Washington - Fall 2025}

\begin{document}

	\begin{frame}
		\titlepage
	\end{frame}

	\begin{frame}[fragile]
		\frametitle{Themes}

		Theme is used to change the non-data elements of the plot:

		\vspace{1em}

		\begin{center}
		\begin{tabular}{|l|l|l|}
		\hline
    		Theme & Type & Arguments \\ 
		\hline
		axis.title.x & element\_text & size, color, family, angle \\
		\hline
		axis.title.y & element\_text & size, color, family, angle \\
		\hline
		plot.background & element\_rect & fill, color, linewidth \\
		\hline
		panel.background & element\_rect & fill, fill, color, line width \\
		\hline
		panel.grid.major & element\_line & color, linetype, linewidth \\
		\hline
		\end{tabular}
		\end{center}

		\vspace{1em}

		Type \verb|?theme| to show all possible types of themes, their types and their arguments.

	\end{frame}

	\begin{frame}[fragile]
		\frametitle{Themes}

		You can add themes to the plot to customize the non-data elements:

		\vspace{1em}

		\begin{exampleblock}{}
		\begin{BVerbatim}
p1 <- ggplot(dfn, aes(Genre, WorldGross)) 
p2 <- p1+ geom_bar(aes(fill=LeadStudio), 
                       stat="Identity",
                       position="dodge")
p3 <- p2 + theme(axis.title.x=element_text(size=15),
                 axis.title.y=element_text(size=15),
plot.background=element_rect(fill="gray87"),
panel.background=element_rect(fill="beige"),
panel.grid.major=element_line(color="Gray",
                              linetype=1))
		\end{BVerbatim}
		\end{exampleblock}{}

	\end{frame}

	\begin{frame}[fragile]
		\frametitle{Themes}

		You can use predefined themes:

		\vspace{2em}

		\begin{exampleblock}{}
		\begin{BVerbatim}
p2 + theme_bw() + ggtitle("theme_bw()")
p2 + theme_classic() + ggtitle("theme_classic()")
p2 + theme_classic() + ggtitle("theme_gray()")
p2 + theme_minimal() + ggtitle("theme_minimal()")
		\end{BVerbatim}
		\end{exampleblock}{}

	\end{frame}

	\begin{frame}[fragile]
		\frametitle{Themes}

		You can also use define your own theme:

		\begin{exampleblock}{}
		\begin{BVerbatim}
mytheme <- theme(legend.title=element_blank(),
           legend.position="bottom",
           text=element_text(color="Blue"),
           axis.text=element_text(size=12,
                                  color="Red"),
           axis.title=element_text(size=rel(1.5)))
		\end{BVerbatim}
		\end{exampleblock}{}

		and use it for a single plot:

		\begin{exampleblock}{}
		\begin{BVerbatim}
p2 + mytheme + ggtitle("Changed Plot with my theme")
		\end{BVerbatim}
		\end{exampleblock}{}

		or for all the plots by placing it at the beginning of your code:

		\begin{exampleblock}{}
		\begin{BVerbatim}
theme_set(my_theme)
		\end{BVerbatim}
		\end{exampleblock}{}

	\end{frame}

	\begin{frame}[fragile]
		\frametitle{Themes}

		You can change the color palette.

		\vspace{2em}

		Type \verb|?scale_fill_brewer| to see all the color palettes available.

		\vspace{2em}

		\begin{exampleblock}{}
		\begin{BVerbatim}
p4 <- p2 + theme_bw() + ggtitle("theme_bw()")
p4 + scale_fill_brewer(palette="Spectral")
		\end{BVerbatim}
		\end{exampleblock}{}

	\end{frame}

	\begin{frame}[fragile]
		\frametitle{Bubble charts}

		You can make the size of each marker proportional to a variable:

		\vspace{2em}

		\begin{exampleblock}{}
		\begin{BVerbatim}
ggplot(dfs, aes(x=Year,
                y=Electricity_consumption_per_capita))
+ geom_point(aes(size=population, color=Country))
+ scale_size(breaks=c(0, 1e+8, 0.3e+9, 0.5e+9,
                      1e+9, 1.5e+9), range=c(1, 5))
		\end{BVerbatim}
		\end{exampleblock}{}

	\end{frame}

	\begin{frame}[fragile]
		\frametitle{Density plots}

		Instead of plotting an histogram, ggplot2 will estimate the density before plotting it:

		\vspace{1em}

		\begin{exampleblock}{}
		\begin{BVerbatim}
ggplot(df, aes(x=loan_amnt)) + 
geom_density() + 
facet_wrap(~grade)
		\end{BVerbatim}
		\end{exampleblock}{}

		\vspace{1em}

		It is also possible to superimpose the density plots:

		\vspace{1em}
	
		\begin{exampleblock}{}
		\begin{BVerbatim}
ggplot(df, aes(x=loan_amnt)) + 
geom_density(aes(fill=grade), alpha=1/2) +
scale_fill_brewer(palette="Dark2")
		\end{BVerbatim}
		\end{exampleblock}{}

	\end{frame}

	\begin{frame}[fragile]
		\frametitle{Time series plots}

		Use\verb|geom_line| to make time series plots:

		\vspace{2em}

		\begin{exampleblock}{}
		\begin{BVerbatim}
df_fb$Date <- as.Date(df_fb$Date)
ggplot(df_fb, aes(x=Date, y=Close, group=1)) + 
geom_line(color="black", na.rm=TRUE) +
scale_x_date(date_breaks='3 month')
		\end{BVerbatim}
		\end{exampleblock}{}

		\vspace{2em}

		We use \verb|group=1| one there is only one line to be show.

		\vspace{2em}

		Use \verb|group=MyCategory| to plot a line per value of a categorical variable.

	\end{frame}

	\begin{frame}[fragile]
		\frametitle{Statistical summaries}

		You can add statistical summaries of the data (e.g. mean, median, quantiles) to the plot:

		\vspace{2em}
	
		\begin{exampleblock}{}
		\begin{BVerbatim}
ggplot(df_fb, aes(Month, Close)) + 
geom_point(color="red", alpha=1/2,
           position=position_jitter(h=0.0, w=0.0)) +
stat_summary(geom="line", fun="quantile",
             fun.args=list(probs=.1), linetype=2,
             color="green", size=1) +
stat_summary(geom="line", fun="quantile",
             fun.args=list(probs=.9), linetype=2,
             color="green", size=1)
		\end{BVerbatim}
		\end{exampleblock}{}

	\end{frame}

	\begin{frame}[fragile]
		\frametitle{Linear regression}

		Ggplot can also fit a linear regression to the data and add it to the plot. It may be done for all the data:

		\vspace{2em}
	
		\begin{exampleblock}{}
		\begin{BVerbatim}
ggplot(dfs, aes(gdp_per_capita,
                Electricity_consumption_per_capita)) +
geom_point(aes(color=Country)) +
stat_smooth(method=lm)
		\end{BVerbatim}
		\end{exampleblock}{}

	\end{frame}

	\begin{frame}[fragile]
		\frametitle{Linear regression}

		Or you can fit a linear regression to each category:

		\vspace{2em}
	
		\begin{exampleblock}{}
		\begin{BVerbatim}
ggplot(dfs, aes(gdp_per_capita,
                 Electricity_consumption_per_capita,
                 color=Country)) +
geom_point() +
stat_smooth(method=lm)
		\end{BVerbatim}
		\end{exampleblock}{}

	\end{frame}

	\begin{frame}[fragile]
		\frametitle{Correlations}

		You can also plot correlations between different variables:

		\vspace{2em}
	
		\begin{exampleblock}{}
		\begin{BVerbatim}
M <- cor(df)
corrplot(M, method="number")
corrplot(M, method="pie")
corrplot(M, method="ellipse")
		\end{BVerbatim}
		\end{exampleblock}{}

	\end{frame}

	\begin{frame}[fragile]
		\frametitle{Maps}

		You can use maps from the \verb|maps| and \verb|map_data| packages to plot maps:

		\vspace{1em}
	
		\begin{exampleblock}{}
		\begin{BVerbatim}
world_map <- map_data("world")
states_map <- map_data("state")
		\end{BVerbatim}
		\end{exampleblock}{}

		\vspace{1em}

		You can select only some polygons from the map using:

		\vspace{1em}

		\begin{exampleblock}{}
		\begin{BVerbatim}
europe <- map_data("world",
region=c("Germany", "Spain", "Italy",
         "France", "UK", "Ireland")) 
		\end{BVerbatim}
		\end{exampleblock}{}

	\end{frame}

	\begin{frame}[fragile]
		\frametitle{Maps}

		To plot the data, we may use \verb|geom_polygom| or \verb|geom_path|.

		\vspace{2em}

		The projection is defined with \verb|coord_map|.

		\vspace{2em}
	
		\begin{exampleblock}{}
		\begin{BVerbatim}
ggplot(states_map, aes(x=long, y=lat, group=group)) +
geom_polygon(fill="white", color="black")
		\end{BVerbatim}
		\end{exampleblock}{}

		\vspace{2em}

		\begin{exampleblock}{}
		\begin{BVerbatim}
ggplot(states_map, aes(x=long, y=lat, group=group)) +
geom_path() +
coord_map("mercator")
		\end{BVerbatim}
		\end{exampleblock}{}

	\end{frame}
\end{document}
