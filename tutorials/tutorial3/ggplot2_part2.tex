% !TEX encoding = UTF-8 Unicode
\documentclass{beamer}

\usepackage{color}
\usepackage{fancyvrb}
\usepackage{gensymb}
\usepackage{hyperref}
\usepackage{textcomp}
\usepackage{tikz}

\usetikzlibrary{arrows,positioning,shapes,shapes.multipart} 
\tikzstyle{every picture}+=[remember picture]

\definecolor{mygreen}{rgb}{0.88,0.95,0.88}

\usetheme{Warsaw}

\title[Data Visualization with R - Ggplot2 tutorial (part 2)]{Data Visualization with R \\ Ggplot2 tutorial (part 2)}
\author{Ariane Ducellier}
\date{University of Washington - Fall 2025}

\begin{document}

	\begin{frame}
		\titlepage
	\end{frame}

	\begin{frame}[fragile]
		\frametitle{Themes}

		Theme is used to change the non-data elements of the plot:

		\vspace{1em}

		\begin{center}
		\begin{tabular}{|l|l|l|}
		\hline
    		Theme & Type & Arguments \\ 
		\hline
		axis.title.x & element\_text & size, color, family, angle \\
		\hline
		axis.title.y & element\_text & size, color, family, angle \\
		\hline
		plot.background & element\_rect & fill, color, linewidth \\
		\hline
		panel.background & element\_rect & fill, fill, color, line width \\
		\hline
		panel.grid.major & element\_line & color, linetype, linewidth \\
		\hline
		\end{tabular}
		\end{center}

		\vspace{1em}

		Type \verb|?theme| to show all possible types of themes, their types and their arguments.

	\end{frame}

	\begin{frame}[fragile]
		\frametitle{Themes}

		You can add themes to the plot to customize the non-data elements:

		\vspace{1em}

		\begin{exampleblock}{}
		\begin{BVerbatim}
p1 <- ggplot(dfn, aes(Genre, WorldGross)) 
p2 <- p1+ geom_bar(aes(fill=LeadStudio), 
                       stat="Identity",
                       position="dodge")
p3 <- p2 + theme(axis.title.x=element_text(size=15),
                 axis.title.y=element_text(size=15),
plot.background=element_rect(fill="gray87"),
panel.background=element_rect(fill="beige"),
panel.grid.major=element_line(color="Gray",
                              linetype=1))
		\end{BVerbatim}
		\end{exampleblock}{}

	\end{frame}

	\begin{frame}[fragile]
		\frametitle{Themes}

		You can use predefined themes:

		\vspace{2em}

		\begin{exampleblock}{}
		\begin{BVerbatim}
p2 + theme_bw() + ggtitle("theme_bw()")
p2 + theme_classic() + ggtitle("theme_classic()")
p2 + theme_classic() + ggtitle("theme_gray()")
p2 + theme_minimal() + ggtitle("theme_minimal()")
		\end{BVerbatim}
		\end{exampleblock}{}

	\end{frame}

	\begin{frame}[fragile]
		\frametitle{Themes}

		You can also use define your own theme:

		\begin{exampleblock}{}
		\begin{BVerbatim}
mytheme <- theme(legend.title=element_blank(),
           legend.position="bottom",
           text=element_text(color="Blue"),
           axis.text=element_text(size=12,
                                  color="Red"),
           axis.title=element_text(size=rel(1.5)))
		\end{BVerbatim}
		\end{exampleblock}{}

		and use it for a single plot:

		\begin{exampleblock}{}
		\begin{BVerbatim}
p2 + mytheme + ggtitle("Changed Plot with my theme")
		\end{BVerbatim}
		\end{exampleblock}{}

		or for all the plots by placing it at the beginning of your code:

		\begin{exampleblock}{}
		\begin{BVerbatim}
theme_set(my_theme)
		\end{BVerbatim}
		\end{exampleblock}{}

	\end{frame}

	\begin{frame}[fragile]
		\frametitle{Themes}

		You can change the color palette.

		\vspace{2em}

		Type \verb|?scale_fill_brewer| to see all the color palettes available.

		\vspace{2em}

		\begin{exampleblock}{}
		\begin{BVerbatim}
p4 <- p2 + theme_bw() + ggtitle("theme_bw()")
p4 + scale_fill_brewer(palette="Spectral")
		\end{BVerbatim}
		\end{exampleblock}{}

	\end{frame}

\end{document}
