% !TEX encoding = UTF-8 Unicode
\documentclass{beamer}

\usepackage{color}
\usepackage{gensymb}
\usepackage{hyperref}
\usepackage{textcomp}

\usetheme{Warsaw}

\title[Instructions for the assignments]{Data Visualization with R - Instructions for the assignments}
\author{Ariane Ducellier}
%\date{}

\begin{document}

	\begin{frame}
		\titlepage
	\end{frame}

	\begin{frame}


Example of data

data() in R

Add links to documentation on these datasets for reference.



For Assignment 1

Choose the dataset, what kind of variable is it, what type of data in R (fill in Table like in page 13)

https://data.un.org/

https://data.gov/

https://data.oecd.org/

https://www.google.com/publicdata/directory

The evaluation of the project involves four stages:

Draft number 1 (4%), due 04/21:  A two-page summary describing a substantive question you are interested in, an identifying data sources that can be used to address this it.  You are free to pick your own dataset and questions.  If you need some hints, a good source of ideas is

 http://hackforchange.org/challenges

(but note that just a small number of projects in this website are a fit for this class).  Some interesting datasets can be found at 

http://www.data.gov 

and

https://www.google.com/publicdata/directory

but you will need to think of questions that can be answered with these datasets (typically a worse approach than to start with a problem and finding data to illuminate it).   The book by Alberto Cairo “The Functional Art” (which is one of our main textbooks) has a number of good examples. 
Draft number 2 (8%), due 05/14:  At this stage, you are expected to provide five to six (preliminary) visualizations for their datasets, along with a brief text with an analysis of each one of them (one paragraph for each visualization).  The instructor will provide feedback to help improve the visualizations.  Although story flow is not a critical consideration at this point, you should start paying attention to it.
Draft number 3 (4%), due 05/28:  You are expected to provide a draft of the layout of your interactive visualization.  At this point, it is critical to focus on the overall presentation as well as in story flow and thematic coherence:  What is the punch line behind your infographic/interactive visualization?  A title for the project should also be included.
Final presentation (14%) due 06/10:  In lieu of the traditional public exhibition, you will be posting your final project to Tableau Online

https://us-west-2b.online.tableau.com/#/site/stat080bspring2020/home

The workbook is due by 8:00 am. In addition to submitting the final project, you will need to complete a self-evaluation and vote for your favorite project in order to receive full credit.

	\end{frame}


\end{document}
